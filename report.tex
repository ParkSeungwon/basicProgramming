\documentclass[11pt,a4paper]{article}
\synctex=1
\usepackage[utf8]{inputenc}
\usepackage[margin=1cm]{geometry}
\usepackage{graphicx}
%\usepackage{verbatim}
\usepackage{listings}
\usepackage{textcomp}
\usepackage{courier}
\usepackage{libertine}
\usepackage{pgfornament}
\usepackage{eso-pic}
\usepackage[hangul]{kotex}
\linespread{1.3}

\title{
	\centering
	\pgfornament[width=12cm,color=teal]{84}\\
	\vspace{1cm}
	\fontsize{50}{50} \selectfont {기초 프로그래밍 실습\\11월 4주차}\\
		\pgfornament[width=12cm,color=teal]{88}\\
	\vfill}
\author{
	\LARGE
	\begin{tabular}{rl}
		\hline
		학번 : & 2016110056\\ 
		학과 : & 불교학부 \\
		이름 : & 박승원\\
		날짜 : & \today\\
		\hline
	\end{tabular}\vspace{2cm}
	\\
\includegraphics[width=0.5\textwidth]{logo.jpg}
	}
\date{}


\begin{document}
\maketitle
\pagenumbering{gobble}
\noindent
\lstset{language=C++, columns=flexible, tabsize=4, frame=shadowbox, showstringspaces=false, breaklines=true, upquote=true, basicstyle=\normalsize}
\begin{enumerate}
7. 직원들의 기본급이 배열 A[]에 저장되어 있다. 배열 B[]에는 직원들의 보너스가 저장되어 있다. 기본급과 보너스를 합하여 이번 달에 지급할 월급의 총액을 계산하고자 한다. A[]와 B[]를 더하여 배열 C[]에 저장하는 함수를 작성하고 테스트하라. 즉, 모든 i에 대하여 C[i] = A[i] + B[i]가 된다.
\lstinputlisting{7.c}
\includegraphics[width=\textwidth]{7.png}

8. 직원들의 월급이 배열 A[]에 저장되어 있다고 가정하자. 이번 달에 회사에서 지급할 월급의 총액을 계산하고자 한다. 정수형 배열 원소들의 합을 구하여 반환하는 함수를 작성하고 테스트하라.
\lstinputlisting{8.c}
\includegraphics[width=\textwidth]{8.png}

9. 직원들의 월급이 저장된 배열에서 월급이 200만원인 사람을 찾고 싶을 때가 있다. 주어진 값을 배열 A[]에서 탐색하여 배열 원소의 인덱스를 반환하늖 함수를 작성하고 테스트하라.
\lstinputlisting{9.c}
\includegraphics[width=\textwidth]{9.png}

10.2개의 정수를 입력받아서 최대 공약수와 최소 공배수를 반환하는 함수를 작성하고 테스트하라. 
최대 공약수는 유클리드의 방법을 사용하여서 계산한다.

유클리드 호제법(- 互除法, Euclidean algorithm)은 2개의 자연수 또는 정식(整式)의 최대공약수를 구하는 알고리즘의 하나이다. 
호제법이란 말은 두 수가 서로(互) 상대방 수를 나누어(除)서 결국 원하는 수를 얻는 알고리즘을 나타낸다. 2개의 자연수(또는 정식) a, b에 대해서 a를 b로 나눈 나머지를 r이라 하면(단, a>b), a와 b의 최대공약수는 b와 r의 최대공약수와 같다. 이 성질에 따라, b를 r로 나눈 나머지 r'를 구하고, 다시 r을 r'로 나눈 나머지를 구하는 과정을 반복하여 나머지가 0이 되었을 때 나누는 수가 a와 b의 최대공약수이다.
\lstinputlisting{10.c}
\includegraphics[width=\textwidth]{10.png}

11. 2개의 정렬된 정수 배열 A[]와 B[]는 똑같은 크기로 정의되어 있다고 가정한다. 
배열 C[]에는 충분한 공간이 확보되어 있다고 가정하자. 합치는 알고리즘은 다음과 같다. 
먼저 A[0]와 B[0]를 비교한다. 
만약 A[0]가 B[0]보다 작으면 A[0]를 C[0]에 복사한다.
다음에는 A[1]과 B[0]를 비교한다.
이번에는 B[0]가 A[1]보다 작다면 B[0]를 C[1]에 저장한다. 
똑같은 방식으로 남아있는 원소들을 비교하여 더 작은 원소를 C[]로 복사한다. 
만약 A[]나 B[] 중에서 어느 하나가 먼저 끝나게 되면 남아있는 원소들을 전부 C[]로 이동한다. 
다음의 그림을 참고하라.
\lstinputlisting{11.c}
\includegraphics[width=\textwidth]{11.png}

\end{enumerate}

\end{document}
