\documentclass[12pt,a4paper]{article}
\synctex=1
\usepackage[utf8]{inputenc}
\usepackage[margin=1cm]{geometry}
\usepackage{graphicx}
%\usepackage{verbatim}
\usepackage{listings}
\usepackage{textcomp}
\usepackage{courier}
\usepackage{libertine}
\usepackage{pgfornament}
\usepackage{eso-pic}
\usepackage{amsmath}
\usepackage{amsfonts}
\usepackage{amssymb}
\usepackage[hangul]{kotex}
\linespread{1.3}

\title{
	\centering
	\pgfornament[width=12cm,color=teal]{84}\\
	\vspace{1cm}
	\fontsize{50}{50} \selectfont {컴퓨터 알고리즘과 실습}\\
		\pgfornament[width=12cm,color=teal]{88}\\
	\vfill}
\author{
	\LARGE
	\begin{tabular}{rl}
		\hline
		학번 : & 2016110056\\ 
		학과 : & 불교학부 \\
		이름 : & 박승원\\
		날짜 : & \today\\
		\hline
	\end{tabular}\vspace{2cm}
	\\
\includegraphics[width=0.5\textwidth]{logo.jpg}
	}
\date{}


\begin{document}
\maketitle
\pagenumbering{gobble}
\noindent
\lstset{language=C++, columns=flexible, tabsize=4, frame=shadowbox, showstringspaces=false, breaklines=true, upquote=true, basicstyle=\normalsize}
\includegraphics[page=2, width=\textwidth]{1.pdf}
\lstinputlisting{1.cpp}
\includegraphics[width=\textwidth]{1.png}

\includegraphics[page=3, width=\textwidth]{1.pdf}
\includegraphics[page=4, width=\textwidth]{1.pdf}
\begin{lstlisting}
M <- 11
c <- [5, 3, 1]
ar <- [1,2,1,2,1, 0,0,0,0,0, 0]
FUNCTION fill_table(idx) :
    m <- min(ar[idx-1], ar[idx-3])
    n <- min(m, ar[idx-5])
    RETURN n+1
ENDFUNCTION

for i in range(5, 11):
    ar[i] <- fill_table(i)
ENDFOR
OUTPUT ar
\end{lstlisting}
함수에서 2번의 비교와 2번의 대입이 이루어지고 각 요소에 한번씩 함수가 호출되므로,
T(4n) = O(n)이다.\\

\begin{tabular}{|c|c|c|c|c|c|c|c|c|c|c|}
	\hline
	1&2&1&2&1& &&&&&\\
	\hline
	1&2&1&2&1&2&&&&&\\
	\hline
	1&2&1&2&1&2&3&&&& \\
	\hline
	1&2&1&2&1&2&3&2&&& \\
	\hline
	1&2&1&2&1&2&3&2&3&& \\
	\hline
	1&2&1&2&1&2&3&2&3&2& \\
	\hline
	1&2&1&2&1&2&3&2&3&2&3 \\
	\hline
\end{tabular}
\vspace{1cm}

\lstinputlisting[language=python]{coin.py}
\includegraphics[width=\textwidth]{coin.png}

\includegraphics[page=5, width=0.97\textwidth]{1.pdf}

\includegraphics[page=6, width=0.97\textwidth]{1.pdf}
누가 먼저 시작하는지에 따라 승패가 갈린다. \\
테이블 상에서 좌측 혹은 위쪽 혹은 좌상에 L이 있으면 승리할 수 있다. 좌에서 하나를 빼거나 우에서 하나를 빼거나 양쪽에서 하나를 빼는 경우가 상대의 패가 되는 경우이므로.\\
\begin{tabular}{|c|c|c|c|c|c|c|}
	\hline
	&0&1&2&3&4&5\\
	\hline
	0&L&W&L&W&L&W\\
	\hline
	1&W&W&W&W&W&W\\
	\hline
	2&L&W&L&W&L&W\\
	\hline
	3&W&W&W&W&W&W\\
	\hline
	4&L&W&L&W&L&W\\
	\hline
	5&W&W&W&W&W&W\\
	\hline
\end{tabular}
\vspace{1cm}
\begin{lstlisting}
	m <- [['L']*11 for i in range(11)]
	ENDFOR
	for i in range(5):
	m[2*i+1][0] <- 'W'
	m[0][2*i+1]='W'
	ENDFOR
	for x in range(1,11):
	for y in range(1,11):
	IF m[x][y-1] is 'L' OR m[x-1][y] is 'L' OR m[x-1][y-1] is 'L':
	m[x][y] <- 'W'
	ENDIF
	ENDFOR
	ENDFOR
	for a in m:
	OUTPUT a;
\end{lstlisting}
\lstinputlisting[language=python]{game.py}
\includegraphics[width=\textwidth]{game.png}
위의 도표 상에서 보면 승리한다.

\includegraphics[page=7, width=\textwidth]{1.pdf}
\includegraphics[page=8, width=\textwidth]{1.pdf}
V는 바이러스의 수, m은 분, B는 박테리아의 수.\\
$V_m=2^m\\
B_{m+1} = (B_m - 2^{m}) \times 2\\
B_{m+1} = 2B_m-2^{m+1}\\
\displaystyle\frac{B_{m+1}}{2^{m+1}}=\frac{B^m}{2^m}-1\\
b_m = \frac{B_m}{2^m}\\
b_0 = n, b_m = n - m\\
\therefore B_m = 2^m(n-m)=0
$
이므로 n분 이후에 박테리아의 수는 0이 된다.



\end{document}
