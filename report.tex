\documentclass[12pt,a4paper]{article}
\synctex=1
\usepackage[utf8]{inputenc}
\usepackage[margin=1cm]{geometry}
\usepackage{graphicx}
%\usepackage{verbatim}
\usepackage{listings}
\usepackage{textcomp}
\usepackage{courier}
\usepackage{libertine}
\usepackage{pgfornament}
\usepackage{eso-pic}
\usepackage[hangul]{kotex}
\linespread{1.3}

\title{
	\centering
	\pgfornament[width=12cm,color=teal]{84}\\
	\vspace{1cm}
	\fontsize{50}{50} \selectfont {컴퓨터 알고리즘과 실습\\3월 2주차}\\
		\pgfornament[width=12cm,color=teal]{88}\\
	\vfill}
\author{
	\LARGE
	\begin{tabular}{rl}
		\hline
		학번 : & 2016110056\\ 
		학과 : & 불교학부 \\
		이름 : & 박승원\\
		날짜 : & \today\\
		\hline
	\end{tabular}\vspace{2cm}
	\\
\includegraphics[width=0.5\textwidth]{logo.jpg}
	}
\date{}


\begin{document}
\maketitle
\pagenumbering{gobble}
\noindent
\lstset{language=C++, columns=flexible, tabsize=4, frame=shadowbox, showstringspaces=false, breaklines=true, upquote=true, basicstyle=\normalsize}
\includegraphics[page=3, width=\textwidth]{1.pdf}
\begin{enumerate}
\item 모든 합을 구한뒤, 1부터 N까지의 합에서 뺀다.
\item 
	i 1 to N : sum += ar[i] \\
	return N(N+1)/2 - sum
\item
\lstinputlisting{1.cpp}	
\item
\includegraphics[width=\textwidth]{1.png}	
\end{enumerate}






\includegraphics[page=4, width=\textwidth]{1.pdf}
\begin{enumerate}
	\item 
	while M \textless 0 :\\
		for i 0 to N :\\
			while M\textgreater c[i] : M -= c[i], coin++\\
\item
\lstinputlisting{2.cpp}	
\includegraphics[width=\textwidth]{2.png}	
\item
	select a coin\\
	M -= coin\\
	if M != 0 : repeat selecting

\item 
\lstinputlisting{3.cpp}	
\includegraphics[width=\textwidth]{3.png}	

\item
	15, 왜냐하면 한 코인의 가치의 두배를 만족시키는 자신보다 높은 값의 코인과 작은 값의 코인의 조합이 존재해야 한다.
\item
\lstinputlisting{6.cpp}	
\includegraphics[width=\textwidth]{6.png}	
모든 경우에 동일하므로 아무런 출력이 발생하지 않았다.

\end{enumerate}


\end{document}
