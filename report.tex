\documentclass[12pt,a4paper]{article}
\synctex=1
\usepackage[utf8]{inputenc}
\usepackage[margin=1cm]{geometry}
\usepackage{graphicx}
%\usepackage{verbatim}
\usepackage{listings}
\usepackage{textcomp}
\usepackage{courier}
\usepackage{libertine}
\usepackage{pgfornament}
\usepackage{eso-pic}
\usepackage{amsmath}
\usepackage{amsfonts}
\usepackage{amssymb}
\usepackage[hangul]{kotex}
\linespread{1.3}

\title{
	\centering
	\pgfornament[width=12cm,color=teal]{84}\\
	\vspace{1cm}
	\fontsize{50}{50} \selectfont {컴퓨터 알고리즘과 실습}\\
		\pgfornament[width=12cm,color=teal]{88}\\
	\vfill}
\author{
	\LARGE
	\begin{tabular}{rl}
		\hline
		학번 : & 2016110056\\ 
		학과 : & 불교학부 \\
		이름 : & 박승원\\
		날짜 : & \today\\
		\hline
	\end{tabular}\vspace{2cm}
	\\
\includegraphics[width=0.5\textwidth]{logo.jpg}
	}
\date{}


\begin{document}
\maketitle
\pagenumbering{gobble}
\noindent
\lstset{language=C++, columns=flexible, tabsize=4, frame=shadowbox, showstringspaces=false, breaklines=true, upquote=true, basicstyle=\normalsize}
\includegraphics[page=2, width=0.9\textwidth]{2.pdf}

\includegraphics[page=3, width=0.9\textwidth]{2.pdf}
\newpage
\lstinputlisting{1.cpp}
\includegraphics[width=0.9\textwidth]{1.png}
\newpage
\includegraphics[page=4, width=\textwidth]{1.pdf}
\lstinputlisting{3.cpp}
프로그램에서 설정한대로 에지를 설정했을 경우, 해밀턴은 한 경로가 존재하고,
오일러에서는 결과에서 보여지듯이 두 가지의 경로가 존재한다.

해밀턴:AGTAAACTTT\\
오일러:AACTTAAGTT, AAGTTAACT

오일러에서 결과화면에 보이는 $<$AA,AC$>$뒤의 숫자가 순서이다.
프로그램에서 쓰인 그래프 헤더는 맨 마지막에 첨부하였다.

\includegraphics[width=\textwidth]{3.png}

\begin{lstlisting}
	void euler() {//for directed graph
		clearv();
		Vertex<T>* p;
		int edge_count = 0;
		for(p = root; p; p = p->vertex) {
			for(Edge<T>* e = p->edge; e; e = e->edge) {
				p->v++;//out grade
				e->vertex->v--;//in grade
				edge_count++;
			}
		}
		for(p = root; p; p = p->vertex) if(p->v == 1) break;//starting point
		euler_visit(p, 0, edge_count);
	}
	
	void euler_visit(Vertex<T>* p, int n, int edge_count) {
		if(n == edge_count) view();//if all the edges are visited
		for(Edge<T>* e = p->edge; e; e = e->edge) {
			if(!e->v) {//if not visited
				e->v = 1;//mark visited
				e->weight = n+1;//use weight to show visit order
				euler_visit(e->vertex, n+1, edge_count);
				e->v = 0;//remove visited mark
			}
		}
	}
\end{lstlisting}
\includegraphics[page=5, width=0.97\textwidth]{1.pdf}
30 * 3e+10 * 3e+09 / 1e+07 / 60 / 60 /24 /365\\
약800만년 

\includegraphics[page=6, width=0.97\textwidth]{1.pdf}
3e+10/4**10 * 3e+9 /1e+7 /60 /60 /24 / 365\\
약 100일

3e+10/4**15 * 3e+9 /1e+7 /60 /60 /24 \\
약 2.3시간

\lstinputlisting[caption=tgraph.h]{tgraph.h}


\end{document}
