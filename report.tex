\documentclass[11pt,a4paper]{article}
\synctex=1
\usepackage[utf8]{inputenc}
\usepackage[margin=1cm]{geometry}
\usepackage{graphicx}
%\usepackage{verbatim}
\usepackage{listings}
\usepackage{textcomp}
\usepackage{courier}
\usepackage[hangul]{kotex}
\linespread{1.3}

\begin{document}
\pagenumbering{gobble}
\begin{center}
	\Huge기초프로그래밍 실습 과제\\
	\vspace{2cm}
\hfill\includegraphics[height=30pt]{logo.jpg}

\hfill\Large 2016110056 불교학부 박승원

\hfill\today
\end{center}

\noindent
\lstset{language=C++, columns=flexible, tabsize=4, frame=shadowbox, showstringspaces=false, breaklines=true, upquote=true, basicstyle=\ttfamily\large}
\begin{enumerate}
\includegraphics[page=3, width=\textwidth]{1.pdf}
\lstinputlisting{1.c}	
\includegraphics[width=\textwidth]{1.png}	

\includegraphics[page=4, width=\textwidth]{1.pdf}
\lstinputlisting{2.c}	
\includegraphics[width=\textwidth]{2.png}	


\includegraphics[page=5, width=\textwidth]{1.pdf}
\lstinputlisting{3.c}	
\includegraphics[width=\textwidth]{3.png}	

\includegraphics[page=6, width=\textwidth]{1.pdf}
\lstinputlisting{4.c}	
\includegraphics[width=\textwidth]{4.png}	


\includegraphics[page=7, width=\textwidth]{1.pdf}
\lstinputlisting{5.c}	
\includegraphics[width=\textwidth]{5.png}	


\includegraphics[page=8, width=\textwidth]{1.pdf}
\lstinputlisting{6.c}

문자열도 컴퓨터에서는 숫자로 표현되며, 그에 대한 표현양식을 어떻게 정해주는가에 따라서 문자로도 혹은 숫자로도 표현된다는 것을 알 수 있다.

\includegraphics[width=\textwidth]{6.png}	
\lstinputlisting{7.c}	
\includegraphics[width=\textwidth]{7.png}	
\end{enumerate}

\end{document}
