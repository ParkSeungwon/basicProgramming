\documentclass[12pt,a4paper]{article}
\synctex=1
\usepackage[utf8]{inputenc}
\usepackage[margin=1cm]{geometry}
\usepackage{graphicx}
%\usepackage{verbatim}
\usepackage{listings}
\usepackage{textcomp}
\usepackage{courier}
\usepackage{libertine}
\usepackage{pgfornament}
\usepackage{eso-pic}
\usepackage{amsmath}
\usepackage{amsfonts}
\usepackage{amssymb}
\usepackage[hangul]{kotex}
\linespread{1.3}

\title{
	\centering
	\pgfornament[width=12cm,color=teal]{84}\\
	\vspace{1cm}
	\fontsize{50}{50} \selectfont {컴퓨터 알고리즘과 실습}\\
	Gene Reconstruction
		\pgfornament[width=12cm,color=teal]{88}\\
	\vfill}
\author{
	\LARGE
	\begin{tabular}{rl}
		\hline
		학번 : & 2016110056\\ 
		학과 : & 불교학부 \\
		이름 : & 박승원\\
		날짜 : & \today\\
		\hline
	\end{tabular}\vspace{2cm}
	\\
\includegraphics[width=0.5\textwidth]{logo.jpg}
	}
\date{}


\begin{document}
\maketitle
\pagenumbering{gobble}
\noindent
\lstset{language=C++, columns=flexible, tabsize=4, frame=shadowbox, showstringspaces=false, breaklines=true, upquote=true, basicstyle=\normalsize}

Problem Definition

\section{Data preparation}

Make 1 million random characters of ACGT. \textrightarrow think as a human genome sequence.

From above make 1\% of mutation. \textrightarrow A person's genome sequence to produce short reads.

From above randomly cut 48 byte of characters \textrightarrow short reads.

\section{Reconstruction}

\end{document}
